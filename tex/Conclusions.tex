\addcontentsline{toc}{chapter}{Conclusioni}
\chapter*{Conclusioni}

Abbiamo considerato la dinamica di un atomo a due livelli inserito in una cavit\`a, la cui densitta spettrale viene descrita da una Lorentziana. Senza l'uso della teoria delle perturbazioni, deriviamo due equazioni differenziali accoppiate che descrivono il sistema. Sono stati analizzati i limiti quando la ampiezza a met\`a altezza della Lorentziana \`e molto stretta e quando \`e molto larga, i risultati sono stati confrontati con quelli del modello  di Jaynes-Cummings e della teoria di Weisskopf-Wigner del'missione spontanea.

Abbiamo discusso come si pu\`o costruire una misura per la non~markovianit\`a dei processi quantistici nei sistemi aperti in termini di informazione che fluisce dall'ambiente al sistema durante la sua l'evoluzione temporale. 

Abbiamo applicato queste definizione allo studio della non~markovianit\`a di un sistema a due livelli che interagisce con un ambiente strutturato a temperatura zero. 

Nel limite dell'accoppiamento debole cio\`e $g \ll \G$, alla risonanza e per valori moderatamente piccoli di $\delta$, otteniamo $\mathcal{N} = 0$. Mentre per $\delta$ maggiori di $\G$ troviamo che $\mathcal{N} > 0$ per frequenze di detuning al di sopra del valore di soglia. In particolare per $g = 0.1 \G$ la frequenza di soglia \`e data da $\delta^{\ast} = 3.635 \G$ e la coppia di stati iniziali ottimali \`e data da $\md1 (0) = \ket b \bra b$ e $\md2 (0) = \ket e \bra e$

Per una cavit\`a moderatamente buona cio\`e  $g \sim \G$ possiamo distinguere due casi $g < \G$ e $g > \G$. 

Per $g < \G$ la misura di non~markovianit\`a \`e positiva, $ \mathcal{N} > 0 $, per valori di $\delta$ al di sotto e al di sopra di due valori di soglia, mentre una regione di comportamento puramente markoviano si sviluppa per valori intermedi. In particolare per $R = 0.8~\G$ troviamo che $\delta_{1}^{\ast} = 0.283~\G$ e $\delta_{2}^{\ast} = 2.956~\G$.

Per $g > \G$ si osserva che $\mathcal{N}$ \`e sempre maggiore di zero dopo che le due frequenze di soglia collassano in una. In particolare per $g = 1.2~\G$ troviamo che il valore della frequenza di soglia \`e $\delta^{\ast} = 1.438$.

In entrambi i casi la frequenza di soglia segnala un cambiamento della coppia ottimale. La coppia ottimale \`e data da qualsiasi coppia di stati antipodali sull'equatore della sfera di Bloch per frequenze di detuning al di sotto del valore di soglia inferiore $\delta_{1}^{\ast}$. Mentre per valori al di sopra di $\delta_{2}^{\ast}$ \`e data dalla coppia di stati antipodali nord-sud della sfera di Bloch.
